\documentclass{article}

\usepackage{Custom_Latex/Summary_Notes/notes}
\usepackage{array}
\usepackage{amsmath}
\usepackage{mathtools}
\usepackage{graphicx}
\usepackage{amssymb}
\graphicspath{{./Images/}}

\everymath{\displaystyle}
\DeclarePairedDelimiter{\ceil}{\lceil}{\rceil}

\begin{document}
\title{IADS - Summary Notes}
\author{Maksymilian Mozolewski}
\maketitle
\pagebreak
\tableofcontents
\pagebreak
% SEMESTER 1 %
% WEEK 1 %
% TUTORIAL - DONE %
\section{Introduction}
\nDefinition{Algorithm}{
a mathematical abstraction of a computer program. with the programming language represented as pseudocode/structured english and the computer being represented by a \bd{model of computation}}
\subsection{Model of Computation}
\nDefinition{Definition}{
specifies what operations an algorithm is allowed and how much they cost (time, etc)}
\subsubsection{Random access machine}
\nDefinition{Definition}{A simple computation model which uses Random Access memory\par
\bd{memory} - constant time loads, computations and stores}  
\subsubsection{Pointer Machine}
\nDefinition{Definition}{
dynamically allocated objects, an object has a constant number of fields, a field is either a word or a pointer.}
\subsubsection{Python Model}
\nDefinition{Definition}{
"list" = array\par
objects with constant attributes(constant time access)\par
append - constant time\par
concat - $O(m + n)$ time\par
'in' - linear time\par
sort - $O(nlog(n))$\par
dictionary - mostly linear time\par 
long addition - $O(n + m)$\par
long multiplication - $O(m + n)^{log(3)}$\par
}
% DAY - TUESDAY %
% LECTURE - 3 %
% READING - 100% %
% NOTES_COMPLETE - %
\section{Basic Sorting}
\subsection{Insertion Sort}
\subsection{Merge Sort} 
% LECTURE - 3 %
% READING - 100% %
% NOTES_COMPLETE - %
\section{Data Structures}
\subsection{Priority Queue}
\nDefinition{Definition}{
set of elements with their priorities, operations performed are: taking elements out with the highest priority, changing priorities etc}
% SEMESTER 2 %
% WEEK 2 %
% TUTORIAL - DONE %

% DAY - TUESDAY %
% LECTURE - 1 %
% READING - 100% %
% NOTES_COMPLETE - %
\section{Algorithm Techqniques}
\subsection{Divide \& Conquer}
\nDefinition{Definition}{
The technique when we design an algorithm to solve a problem by taking an instance \emph{I} (of size n), then:
\begin{itemize}
    \item divide the problem into k smaller subproblems
    \item Making k recursive calls to obtain the answer for the sub-problems
    \item Take answers from 2. and do some computation to get the overall answer for the original input I.
\end{itemize}

Details will vary (the number k, how to combine answers etc.)\par
in some cases, Divide-and-Conquer can give rise to an efficient (Polynomial time) algorithm - for example Mergesort, Quicksort.\par
\bd{Master Theorem} - often features in the analysis. \par
but not always effective.
}
\subsubsection{Toy Example}
\nDefinition{Fibonacci numbers}{
The Fibonacci numbers are defined as: 
\begin{gather*}
    F_{0} = 0,\\
    F_{1} = 1,\\
    F_{n} = F_{n-1} + F_{n-2}\text{    } (for n \geq 2)
\end{gather*}
There is an immediate recursive algorithm:
\bd{Algorithm Rec-Fib}(n)\par
\bd{if} n = 0 \bd{then}\\
\hspace*{15pt} \bd{return} 0
\par
\begin{equation*}
    T(n) = T(n-1) + T(n-2) + \theta(1) \geq F_{n} =(aprox) (1.6)^{n}
\end{equation*}
}
\nDefinition{Dynamic Approach}{
\bd{Algorithm} Dyn-Fib(n)\\
$F[0] = 0$\\
$F[1] = 1$\\
\bd{for} $i \leftarrow 2$ \bd{to} n \bd{do}\\
\hspace*{15pt} $F[i] \leftarrow F[i-1] + F[i-2]$\\
\bd{return} $F[n]$\\
build from the bottom up. \par
we are turning the recursion upside down. \par\bigskip
running time is \bd{$\theta(n)$}
}
\nDefinition{coin-changing problem}{
in the UK coins have denominations 1p, 2p, 5p, 10p, 20p, 50p, 1£ and £2\par
A frequently-executed task in the retail sector involves taking an input value and calculating a collection of coins (may include duplicates) which will sum to that value. \par
We assume an unlimited supply of coins of each value \par
The \bd{coin problem} is the problem, given an input value v ($v \in \mathbb{N_{0}}$)of calculating a collection of coins (of minimum cardinality) that will sum to v\par
Observation: there is a coin which belongs to the optimal solution, and knowing that: 
\begin{equation*}
    C(v) = 1 + C(v - c_{i})
\end{equation*}}
\nDefinition{Edit distance via Dynamic Programming}{
\bd{Edit Distance} - number of mutations to get from one string to another where mutations can be: substitutions, insertions and deletions}
\subsection{Dynamic Programming}
\nDefinition{}{when solution to problem is found by computing smaller solutions first, we need the solution to be an intance of the problem to be expressible in terms of a recurrence, where the right-hand side contains one or more recursive calls for smaller instances of the same proble. We need to organize storage for all results of all possible subproblems. We need an algorithm to control the order in which subproblems on the right side of the erecurrence are solved in advance of the call to the recurrence on the left side}

% DAY - X %
% LECTURE - 2 %
% READING -  %
% NOTES_COMPLETE -  %
\section{}
\end{document}