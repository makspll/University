\documentclass{article}

\usepackage{notes}
\usepackage{array}
\begin{document}

\title{DMMR Condensed Summary Notes For Quick In-Exam Strategic Fact Deployment }
\author{Maksymilian Mozolewski}
\maketitle
\pagebreak
\nSection{Boolean Logic}
\nDefinition{Equivalence of prepositional statements}{$P \equiv Q$ denotes a logical equivalence between the propositional statements P and Q. Two equivalent propositions have the same truth tables}
\nDefinition{Contrapositive}{let S be a statement of the form $P \rightarrow Q$ then the Contrapositive of S is $\neg Q \rightarrow \neg P$. The Contrapositive of S is logically equivalent to S}
\nTheorem{Boolean Logic Laws}{
\begin{flushleft}
\begin{tabular*}{\textwidth}{lll}
Identity Law: &$P \land T \equiv P$ & $P \lor F \equiv P$\\
Domination Law: &$P \land F \equiv F$ & $P \lor T \equiv T$\\
Idempotency Law: &$P \land P \equiv P$ & $P \lor P \equiv P$\\
Double Negation: &$\neg(\neg P) \equiv P$ &  $ $\\
Commutativity Law: &$P \land Q \equiv Q \land P$ & $P \lor Q \equiv Q \lor P$ \\
Associativity law: &$P \land (Q \land R) \equiv (Q \land P) \land R$ & $P \lor (Q \lor R) \equiv (Q \lor P) \lor R$\\
Distributive Law: &$P \land (Q \lor R) \equiv (P \land Q) \lor (P \land R)$ & $P \lor (Q \land R) \equiv (P \lor Q) \land (P \lor R)$\\
De Morgan's Law: & $\neg(P \land Q) \equiv \neg P \land \neg Q$
\end{tabular*}
\end{flushleft}
}
\nTheorem{Negation of Quantifiers}{
\begin{tabular*}{\textwidth}{ll}
$\neg(\forall x(P(x))) \equiv \exists x \neg P(x)$ & $\neg(\exists x P(x)) \equiv \forall x \neg P(x)$
\end{tabular*}
}
\nSection{Proof Techniques}
\nDefinition{Direct Proof}{use existing propositions and rules of inference to prove the given proposition}
\nDefinition{Proof by Contraposition}{just like direct proof but we use prove contraposition of the given proposition}
\nDefinition{Proof by Contradiction}{let P be the proposition to be proven, then assume $\neg P$ is true and show that $\neg P \rightarrow C$ where C is some logical contradiction of an earlier assumption or fact}
\nSection{Induction}
\nDefinition{Normal Induction}{if P(n) is a predicate on $\mathbb{Z}^{+}$ we follow this process:\\
\emph{Base Case:} we prove P(1) is true\\
\emph{Inductive Hypothesis:} we assume $P(k)$ is true and we set to prove $P(k) \rightarrow P(k + 1)$ is true\\
\emph{Inductive Step:} we show that $P(k) \rightarrow P(k+1)$ is true

}

\nDefinition{Strong Induction}{same as normal induction however instead of assuming P(k) is true, we assume $P(1) \land .. P(k)$ is true and show that it being true implies P(k + 1)}
\nSection{Sets}
\nDefinition{Set}{an unordered collection of objects, called members/elements}
\nDefinition{Important Sets}{
$\mathfbb{B} = \{true,false\}$ Boolean values \\
$\mathfbb{N} = \{0,1,2,3,..\}$ Natural numbers \\
$\mathfbb{Z} = \{...,-3,-2,-1,0,1,2,3,...\}$ Integers \\
$\mathfbb{Z}^{+} = \{z \in \mathfbb{Z} | z > 0\}$ Positive Integers \\
$\mathfbb{R}$ Real Numbers\\ 
$\mathfbb{R}^{+} = \{r\in R | r > 0\}$ Positive Real Numbers \\
$\mathfbb{Q} = \{\frac{a}{b} | a \in \mathbb{Z}, b \in \mathbb{Z}^{+}\}$ Rational Numbers\\
$\mathfbb{Q}^{+} = \{\frac{a}{b} | a \in \mathbb{Z}^{+}, b \in \mathbb{Z}^{+}\}$ Positive Rational Numbers\\
$\mathfbb{C}$ Complex numbers
}
\nDefinition{Power Set}{the power set of a set A consists of all the possible subsets of A including the empty set. if A has n elements, then P(A) will contain $2^{n}$ elements}
\nDefinition{Complement}{the complement of A is the set $\bar{A}$ which contains all the elements which are not in A, relative to the universe of discourse}
\nDefinition{Proofs with sets}{to prove a set is a subset of another, show that an element in the first one must be in the other one, to prove two sets are equal, prove that they are both subsets of each other}
\nTheorem{Set Identities}{
\begin{flushleft}
\begin{tabular*}{\textwidth}{lll}
Identity Law: &$A \cap U = A$ & $A \cup \emptyset = A$\\
Idempotency Law: &$A \cap A = A$ & $A \cup A = A$\\
Commutativity Law: &$A \cap B = B \cap A$ & $A \cup B = B \cup A$\\
De Morgan's Law:&$\overline{A \cap B} = \bar{A} \cap \bar{B}$ & $\overline{A \cup B} = \bar{A} \cup \bar{B}$\\
Absorption Law: &$A \cup (A \cap B) = A$&$A \cap (A \cup B) $\\
Domination Law: &$A \cap \emptyset = \emptyset$&$A \cup U = U$\\
Complementation law: &$ \overline{(\overline{A})} = A$ &$ $\\
Associative Law: &$A \cap (B \cap C) = (A \cap B) \cap C$ & $A \cup (B \cup C) = (A \cup B) \cup C$\\
Distributive Law:& $A \cup (B \cup C) = (A \cup B) \cap (A \cup B)$ &$A \cap (B \cap C) = (A \cap B) \cup (A \cap C)$\\
Complement Law: &$A \cap \overline{A} = \emptyset$&$A \cup \overline{A} = U$\\
\end{tabular*}
\end{flushleft}
}
\nDefinition{Cartesian Product}{$A x B$ is the set of all ordered pairs (a,b) such that $a \in A \land b \in B$}
\nSection{Cardinality}
\nDefinition{Cardinality}{the set A has cardinality $|A|$ which is the number of elements in A}
\nTheorem{Sets comparisons}{for all sets, $|X| \leq |Y|$ iff there is an injection $f: X \rightarrow Y$\\
$|X| = |Y|$ iff there is a bijection $f : X \rightarrow Y$\\
$|X| < |Y|$ iff $|X| \leq |Y| \land |X| \neq |Y|$}
\nDefinition{Countability}{A set S is called countably infinite, iff it has the same cardinality as the positive integers, $|\mathbb{Z}^{+}| = |S|$ we then say it has cardinality $\aleph$\\ A set is called countable iff it is either finite or countably infinite, otherwise it's called uncountable}
\nTheorem{Countability and Union}{if A and B are countable sets, then $A\cup B$ is countable}
\nTheorem{Countability of Big Union}{if I is countable and for each $i \in I$ the set $A_{i}$ is countable then $\bigcup_{i\in I} A_{i}$ is countable}
\nTheorem{Cardinality of Finite Strings}{The set $\sum^{*}$ of all finite strings over a finite alphabet $\sum$ is countably infinite}
\nTheorem{Uncountable sets}{The set of infinite binary strings is uncountable\\The set $[0,1] \subseteq R$ is uncountable\\The set of functions $F = \{f|f : \mathbb{Z} \rightarrow \mathbb{Z}\}$ is uncountable}
\nTheorem{Schroder-Bernstein Theorem}{if $|A| \leq |B| \land |B| \leq |A|$ then $|A| = |B|$}
\nTheorem{Cantor's theorem}{$|A| < |P(A)|$}
\nTheorem{Continuum}{the cardinality of the set $\mathbb{R}$ is $\mathfrak{c}$, and $\aleph < \mathfrak{c}$, there exists an infinite hierarchy of cardinalities of infinite sets}
\nSection{Relations}
\nDefinition{Binary relation}{a binary relation R on sets A and B is a subset $R \subseteq A x B$. e.g. a set of tuples (a,b) with $a \in A \land b \in B$}
\nDefinition{n-ary relation}{given sets $A_{1},...,A_{n}$ a subset $R \subseteq A_{1},...,A_{n}$ is an n-ary relation}
\nTheorem{Relation Union and Intersection}{if $R_{i}$ are relations on $AxB \forall i \in I$ then $\bigcup_{i\in I}R_{i}$ and $\bigcap_{i\in I}R_{i}$ are relations on AxB}
\nDefinition{Relation composition}{let $R \subseteq BxC$ and $S \subseteq A x B$ the composition relation $(R \circ S) \subseteq A x C$ is\\ $\{(a,c) | \exists b(a,b) \in S \land (b,c) \in R\}$}
\nDefinition{Closures}{Closure R is a relation on A: 
\begin{itemize}
    \item $R^{0}$ is the identity relation $(\mathbf{i}_{A})$
    \item $R^{n+1} = R^{n} \circ R$
    \item $R^{*} = \bigcup_{n \geq 0} R^{n}$
\end{itemize}
$R^{*}$ is the reflexive and transitive closure of R}
\nDefinition{Properties of binary relations}{the relation R on A is:\\
reflexive iff $\forall x \in A. (x,x) \in R$\\
symmetric iff $\forall x,y \in A. ((x,y) \in R \rightarrow (y,x) \in R)$\\
antisymmetric iff $\forall x,y \in A (((x,y) \in R \land (y,x) \in R) \rightarrow x = y)$\\
transitive iff $\forall x,y,z \in A(((x,y)\in R \land (y,z) \in R) \rightarrow (x,z) \in R)$}
\nDefinition{Equivalence relations}{A relation R on a set A is an equivalence relation iff it is reflexive, symmetric and transitive}
\nDefinition{Equivalence classes}{let R be an equivalence relation on a set A and $a\in A$. Then\\ $[a]_{R} = \{s|(a,s \in R)\}$\\
is the equivalence class of a w.r.t R}
\nTheorem{Properties on equivalent relations}{let R be an equivalence relation on A and a,b $\in A$ the following three statements are equivalent:
\begin{itemize}
    \item aRb
    \item $[a]_{R} = [b]){r}$
    \item $[a]_{R} \cap [b]){r} \neq \emptyset$
\end{itemize}
}
\nDefinition{Partitions of a set}{A partition of a set A is a collection of disjoint, nonempty subsets that have A as their union.}
\nTheorem{Equivalence To Partition}{If R is an equivalence on A, then the equivalence classes of R form a partition of A. Conversely, given a partition ${A_{i}|i \in I}$ of A there exists an equivalence relation R that has exactly the sets $A_{i}$ for $i \in I$ as its equivalence classes}
\nSection{Functions}
\nDefinition{Function}{let A and B be a non-empty set, then a function f maps exactly one element of B to each element of A. so every element in the function must be defined on all elements in A and there cannot be more than one element of B associated with any element of A}
\nDefinition{function Composition}{if f and g are functions then $f \circ g = f(g(x))$}
\nTheorem{new Functions from old functions}{if f and g are functions then $f + g$ and $f \circ g$ is also a function\\
if f and g are both injective/surjective/bijective then $f\circ g$ is also injective/surjective/bijective (respectively)}
\nDefinition{Injective Functions (one-to-one)}{$f: A\rightarrow B$ is injective iff $\forall a,c \in A (f(a) = f(c) \rightarrow a = c)$}
\nDefinition{Surjective Functions (onto)}{$f : A \rightarrow B$ is surjective iff $\forall b\in B\exists a\in A(f(a) = b)$}
\nDefinition{Bijective Functions (One-to-one correspondence)}{$f : A \rightarrow B$ is a bijection iff it is both injective and surjective}

\nDefinition{Inverse Functions}{let $f : A \rightarrow B$ be a bijection, then the inverse of f, $f^{-1}: B \rightarrow A$ is $f^{-1}(b) = a$ iff $f(a) = b$. So the inverse function exists only for bijections, and it exists for only the elements in B which are mapped to by some element $a \in A$ by f}
\nSection{Sequences}
\nDefinition{Sequences}{Sequences are ordered lists of elements, \\
Example: $f : \mathbb{Z}^{+} \rightarrow \mathbb{Q} f(n) = \frac{1}{n}$ defines the sequence $1,1/2,1/3,1/4,...$ assuming $a_{n} = f(n)$ the sequence is also written $a_{1},a_{2},a_{3}...$ or as $\{a_{n\in\mathbb{Z}^{+}}\}$}
\nDefinition{Sequence over a Set}{a sequence over a set S is a function f from a subset of the integers to the set S. If the domain of f is finite then the sequence is finite.}
\nDefinition{Geometric and Arithmetic progressions}{A geometric progression is a sequence of the form $a, ar, ar^{2},...ar^{n},..$\\
An arithmetic progression is a sequence of the form $a, a + d, a + 2d,...,a + nd$, where the initial elements a, the common ratio r and the common difference d are real numbers}
\nDefinition{Reccurence relations}{A reccurencerelation for $\{a_{n}\}_{n\in \mathbb{N}}$ is an equation that expresses $a_{n}$ in terms of one or more of the elements $a_{0},a_{1},...,a_{n-1}$\\
the initial conditions specify the first elements of the sequence, before the recurrence relation applies\\
A sequence is called a solution of a recurrence relation iff its terms satisfy the recurrence relation}
\nDefinition{Fibonnaci Sequence}{$f(1) = 1, f(2) = 1, f(n) = f(n-1) + f(n-2) for n > 2$}
\nDefinition{Solving recurrence relations with iteration}{$a_{n} = a_{n-1} + 3$ for $n \geq 2$ with $a_{1} = 2$\\ Forward Substitution Method:\\
$a_{2} = 2 + 3$\\
$a_{3} = (2 + 3) + 3 = 2 + 3 \cdot 2$\\
$a_{4} = (2 + 2 \cdot 3) + 3 = 2 + 3 \cdot 3$\\
$a_{n} = a_{n-1} + 3 = (2 + 3 \cdot (2 - 2)) + 3 = 2 + 3 \cdot (n-1)$
}
\nDefinition{Common Sequences}{}
\nSection{Sums}
\nSection{Number Theory}
\nSection{Counting}
\nSection{Graphs}
\nSection{Trees}
\nSection{Discrete Probability}
\nSection{Examples in Probability}
\end{document}
