\documentclass{article}

\usepackage{Custom_Latex/Summary_Notes/notes}
\usepackage{array}
\usepackage{amsmath}
\usepackage{mathtools}
\usepackage{graphicx}
\usepackage{amssymb}
\graphicspath{{./Images/}}

\everymath{\displaystyle}
\DeclarePairedDelimiter{\ceil}{\lceil}{\rceil}

\begin{document}
\title{PWA - Summary Notes}
\author{Maksymilian Mozolewski}
\maketitle
\pagebreak
\tableofcontents
\pagebreak
% WEEK 1 %
% TUTORIAL - DONE %
% HAND-IN - COMPLETE %

% DAY - TUESDAY %
% LECTURE - 1 %
% READING - DONE %
% NOTES_COMPLETE - SAME AS DMMR  %

\section{Introduction}
\subsection{Information}
general information\bigskip\\
new course, restructured.
\section{Counting}
\nTheorem{Product Rule}{if A and B are finite sets then: $|A \times B| = |A| \cdot |B|$}
\nTheorem{General Product Rule}{if $A_{1},A_{2},...,A_{m}$ are finite sets then: $|A_{1},A_{2},...,A_{m}| = |A_{1}|\cdot|A_{2}|\cdots|A_{m}|$}
\nDefinition{Counting Summary}{
\begin{tabular*}{\textwidth}{l|l|l}
    Type & Repetition Allowed ? & Formula\\
    \hline
    r-permutations & No & $\frac{n!}{(n-r)!}$\\[15pt]
    r-combinations & No & $\frac{n!}{r!(n-r)!}$\\[15pt]
    r-permutations & Yes & $n^{r}$\\[15pt]
    r-combinations & Yes & $\frac{(n + r -1)!}{r!(n-1)!}$
\end{tabular*}
}

\nDefinition{Permutations}{A permutation of a set S is an ordered arrangement of the elements
of S.
In other words, it is a sequence containing every element of S exactly
once}

\nTheorem{The Binomial Theorem}{For all $n \geq 0$:
\begin{align*}
    (x + y)^{n} = \sum_{j=0}^{n}{{n \choose j}x^{n-j}y^{j}}
\end{align*}
}

\nTheorem{Multinomial theorem}{for all n $\geq$ 0 and all k $\geq 1$:
\begin{align*}
    ({x_{1}+x_{2}+...+x_{k}})^{n} = \sum_{0\leq n_{1},n_{2},...,n_{k}\leq n}{n\choose n_{1},n_{2},...,n_{k}}x_{1}^{n_{1}}x_{2}^{n_{2}}...x_{k}^{n_{k}}
\end{align*}}

% DAY - THURSDAY %
% LECTURE - 2 %
% READING -  %
% NOTES_COMPLETE -  %
\section{Conditional Probability And Independence}
\subsection{Conditional Probability}
\nDefinition{Conditional Probability}{}
\subsection{Independence of Events}
\subsection{Multiplication rule for probabilities}
\subsection{Law of total probability}
\subsection{Baye's formula}
\subsection{Expectation}
\end{document}